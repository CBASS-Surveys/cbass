\documentclass[10pt,letter]{report}
\usepackage{IEEEtitlepage}

\title{CBASS -- The Constraint Based Anonymizable Survey System}
\subtitle{Specification}
\author{David Carlson, Sam Flint, Kaleb Gaar, Hayden Pollmann}
\company{UNL}
\summary{}
\date{\today}
\version{2.0}

\setcounter{tocdepth}{5}
\setcounter{secnumdepth}{5}

\usepackage{parskip}

\def\TODO{\marginpar{\textbf{TODO}}}
% \def\TODO{}

\usepackage{color}
\usepackage{xcolor}
% \def\hl#1{\colorbox{yellow}{#1}}
\def\hl#1{#1}

% Setup hyperref.  Make sure to change the fields as needed
\usepackage{hyperref}
\hypersetup{
  pdfauthor={David Carlson, Sam Flint, Kaleb Gaar, Hayden Pollmann},
  pdftitle={CBASS -- The Constraint Based Anonymizable Survey System: Specification},
  pdfsubject={Subject of Project},
  pdflang={English},
  pdfkeywords={keywords,comma,seperated},
  colorlinks=true
}

\newenvironment{defns}{
  \begin{description}
    \newcommand{\defn}[2]{\item[##1] ##2}
  }
  {
    \end{description}
  }


% Include a couple of packages for graphic and figures
\usepackage{graphicx}
\usepackage{morefloats}
\usepackage{tabularx}

% Change margins to 0.75 inches
\usepackage[margin=0.75in]{geometry}

\begin{document}

\maketitle
-
\chapter*{Revision History}
\label{sec:revisions}
\addcontentsline{toc}{chapter}{Revision History}


\begin{table}[h]
  \centering
  \begin{tabularx}{\textwidth}{c|X|c|r}
    \textbf{Version}   & \textbf{Changes}                                                                                           & \textbf{Author}         & \textbf{Date} \\
    \hline\hline
    1.0                & Provide initial structure.                                                                                 & Sam Flint               & 2017-09-12    \\
    1.1                & Add Initial Description of main functions.                                                                 & Sam Flint               & 2017-09-13    \\
    1.2                & Added information in overall description                                                                   & Kaleb Gaar              & 2017-09-12    \\
    1.3                & Added sections for sub-functions of main.                                                                  & Sam Flint               & 2017-09-14    \\
    1.4                & Added data collection intro and associated requirements                                                    & David Carlson           & 2017-09-14    \\
    1.5                & Added data retrieval associated requirements                                                               & Hayden Pollmann         & 2017-09-14    \\
    1.6                & Completed description of Survey Designer requirements.                                                     & Sam Flint               & 2017-09-14    \\
    1.7                & Switched ``Data Collection'' and ``Data Retrieval'' around.                                                & Sam Flint               & 2017-09-18    \\
    1.8                & Finished ``Stimulus and Response'' sections in \S\ref{sec:survey-designer} and \S\ref{sec:data-retrieval}. & Sam Flint               & 2017-09-20    \\
    1.9                & Finished ``External Interface Requirements`` sections, added frontend definition                           & Hayden Pollmann         & 2017-09-20    \\
    1.10               & Completed Section 1.                                                                                       & David Carlson           & 2017-09-20    \\
    1.11               & Completed Section 2.                                                                                       & Kaleb Gaar              & 2017-09-20    \\
    1.12               & Completed remaining sections in Section 3, from \S 3.3 to \S 3.6.                                          & Sam Flint               & 2017-09-21    \\
    2.0                & Made edits and polished for version 2.0, including for grammar and consistency.                            & Sam Flint \& Kaleb Gaar & 2017-09-22    \\
    	             % &                                                                                                            &                         &               \\
    \hline
  \end{tabularx}
  \caption{Revision History}
  \label{tab:rev-hist}
\end{table}
\newpage

% Table of Contents
\hypersetup{linkcolor=black}
\tableofcontents
\hypersetup{linkcolor=red}
\newpage

% Optional List of Figures
% \listoffigures
% \newpage

% Optional List of Tables
% \listoftables
% \newpage

\chapter{Introduction}
\label{cha:introduction}

\section{Purpose}
\label{sec:purpose}

This document has been written to define the various functions needed
to design and develop CBASS, the Constraint Based Anonymizable Survey
System.  It is intended to be read by designers (for the purpose of
design), developers (to best understand the design), testers (to
develop complete and accurate tests), and clients (to ensure
specification accuracy).

\section{Scope}
\label{sec:scope}

This document provides the specification for CBASS, the Constraint
Based Anonymizable Survey System.

CBASS will provide a mechanism for researchers who use surveys to gather
and collect data a mechanism to both a) ensure that the data gathered
can be made anonymous; and b) apply certain classes of constraints to
the data collected.  This will be to ensure that the data collected will
always be consistent, and that it will be possible to easily fulfill
the requirements of some human-related research programs.  CBASS will
not, however, provide the ability to perform \emph{any} type of
statistical analysis, but rather will provide a method for a standard,
normalized form of the data to be downloaded.

The classes of constraints possible will be as follows: a) answer X to
question Y forbids answering of question Z; b) answer X to question Y
requires answering question Z; and c) answer X to question Y removes
possible answers n--m from question Z.

\section{Definitions, Acronyms and Abbreviations}
\label{sec:defin-acronyms-abbr}

\begin{defns}
   \defn{Constraints}{Limits such as the permission or refusal of a
     question to be answered within the survey usually based on
     information provided through other questions.}
  \defn{CSV}{Stands for Comma Separated Values, a file which allows
    the saving of data in table-like structures.}
  \defn{Excel}{A Microsoft Excel file which is organized in a table
    for use in many fields such as statistics.}
  \defn{Forbids}{When said of a question constraint, the answer given
    in a question disqualifies the respondent from answering an
    upcoming question.}
  \defn{Frontend}{A graphical user interface.}
  \defn{JSON}{Stands for JavaScript Object Notation, a lightweight
    data interchange format that allows data to be read by humans and
    used by programs easily.}
  \defn{Requires}{When said of a question constraint, the answer given
    in a question necessitates the respondent to answer a question another
    respondent might not need to answer.}
  \defn{Respondents}{Individuals who take part in answering a survey
    created on the system.}
  \defn{Response Value}{A value that denotes what response was chosen,
    used by statisticians for interpretation}
  \defn{Restricts}{When said of a question constraint, the answer given in a
    question disqualifies certain possible responses from appearing
     to a user in an upcoming question, but not disqualifying them
     from answering the question entirely.}
  \defn{SQL}{Stands for Structured Query Language, a standard language
    for relational database management and data manipulation.}
  \defn{User Response Text}{The text displayed to a respondent as a
    question response when he or she is completing the survey.}

\end{defns}

% \section{Reference}
% \label{sec:reference}

\section{Overview}
\label{sec:overview}

The remainder of this document contains an overview of the product,
its functionality, the characteristics of users, and a detailed
description of the interfaces between the software and other services.

\chapter{Overall Description}
\label{cha:overall-description}

\section{Product Perspective}
\label{sec:product-perspective}

\subsection{System Interfaces}
\label{sec:system-interfaces}

The CBASS will be a web tool that uses a database to store
survey information.

\subsection{User Interfaces}
\label{sec:user-interfaces}

A user will navigate to the survey website, and will interact with
it using websites.

\subsection{Communications Interfaces}
\label{sec:communications-interfaces}

To ease the sharing of surveys, the system will optionally access
social media sites to connect survey givers to respondents.

\subsection{Site Adaptation Requirements}
\label{sec:site-adaptation-requrements}

Customization of a small number of visual components of surveys will
be implemented to allow survey creators to represent their site.

\section{Product Functions}
\label{sec:product-functions}

The primary function of the CBASS will be to allow users to create surveys online
to collect information from survey takers. The system will allow constraints to
be placed on questions so that changes in survey questions occur depending on
earlier questions.

\section{User Characteristics}
\label{sec:user-characteristics}

The CBASS will be designed for anyone seeking a constraint based
survey system. However, in this general audience, academic and research based
users are seen as the target users of this system. The implementation of
constraints in surveys will allow researchers to better tailor a particular
survey to their requirements.

\section{Constraints}
\label{sec:constraints}

\subsection{Anonymizing Data}
\label{sec:anonymizing-data}

Since the system will be handling potentially sensitive data, the ability to
make information anonymous is a very important consideration in the design.
Failure to take this into consideration can lead to breach of information and
potential legal issues.

\subsection{Internet Browser Compatibility}
\label{sec:internet-browser-compatibility}

In order to allow users to access the system, considerations must be made to
support most or all modern internet browsers. Failure to make such
considerations may lead to potential users who are unable to use the system.

\section{Assumptions and Dependencies}
\label{sec:assumpt-depend}

This system will be relying on several technologies for its implementation.
These include internet protocols, libraries, and a hosted database. As these
technologies age, there may be need to revise the system to comply with updates to
these technologies.

\chapter{Detailed Component Description}
\label{cha:det-comp-desc}

\section{External Interface Requirements}
\label{sec:extern-interf-requ}

\subsection{User Interfaces}
\label{sec:user-interfaces}

The User will use a frontend that will allow them to navigate through
both the survey creation and the answering of surveys.

\subsection{Software Interfaces}
\label{sec:software-interfaces}

The system will communicate with a web server for hosting of the
application to the internet. The system will also use a SQL database
for housing the survey data.

\subsection{Communications Interfaces}
\label{sec:comm-interf}

The system will interface with the most or all modern web browsers.

\section{Functions}
\label{sec:functions}

\subsection{Survey Designer}
\label{sec:survey-designer}

\subsubsection{Introduction}
\label{sec:introduction-designer}

The ability to create and design surveys, including not only various
questions and the possible answers, but various forms of \hl{constraints},
is vital.  This will be done through the use of a web frontend and will be
composed of several sub-features.

\subsubsection{Stimulus/Response}
\label{sec:stim-resp-designer}

Provided an authenticated User, the User will be able to create a
survey, following a variety of steps, including adding questions,
adding possible responses to questions, adding varying forms of
constraints, and modifying the varying attributes previously
mentioned.

\subsubsection{Associated Requirements}
\label{sec:assoc-req-designer}

\paragraph{Create Survey}
\label{sec:create-survey}

A Survey will be created with the following information initially:
Survey Name, Survey Giver, and Survey Description.  This information
will be stored, a Survey ID returned, and the user prompted to edit
details (\S \ref{sec:edit-survey-details}), add questions (\S
\ref{sec:add-survey-question}), or add constraints (\S
\ref{sec:add-quest-constr}).

\paragraph{Edit Survey Details}
\label{sec:edit-survey-details}

With a Survey selected (by Survey ID), various information related to
the survey will be editable, including theme data (Logo and Colors),
Survey Description, Survey (URL) Short Code, Survey Authentication
Requirements, and various in-survey messages (Survey Start Message,
Survey End Message).  The changed information will be updated and
stored for later use.

\paragraph{Add Survey Question}
\label{sec:add-survey-question}

With a Survey selected (by Survey ID), a user will be able to add a
Question.  There will be three types of responses to select from:
multiple choice, numeric response and free-response.  The type of
Question will be selected, and a Prompt (the Question Text) will be
provided by the user.  This will be stored for later use, a Question
ID returned, and in the event of a multiple-choice question, the user
will be prompted to add possible responses (\S
\ref{sec:add-quest-resp}).

\paragraph{Add Question Response}
\label{sec:add-quest-resp}

With a Question selected (by Question ID), a user will be able to add
possible responses, containing the following information: \hl{User
Response Text} and \hl{Response Value}.  These will be stored for later use.

\paragraph{Add Question Constraint}
\label{sec:add-quest-constr}

With a Survey selected (by Survey ID), a user will be able to create a
question constraint.  As there are three types of constraints, there
are three ways to add constraints: for when an answer \hl{forbids a
question} (\S\ref{sec:answ-forb-quest}), for when an answer \hl{requires a
question} (\S\ref{sec:answ-requ-quest}), and for when an \hl{answer
restricts the possible answers for another question}
(\S\ref{sec:answ-restr-quest}).

\subparagraph{Answer Forbids Question}
\label{sec:answ-forb-quest}

The Forbidding Question and Answer will be selected (first by Question
Text and then by Response Value), and a second question, the Forbidden
Question, will be selected (again, by Question Text).  This
information will then be stored for later use both in the removal of
incomplete responses (\S \ref{sec:remove-responses}) and by constraint
verification logic (\S \ref{sec:verify-constraints}).

\subparagraph{Answer Requires Question}
\label{sec:answ-requ-quest}

The Requiring Question and Answer will be selected (first by Question
Text and then by Response Value), and a second question, the Required
Question will be selected (again, by Question Text).  This information
will then be stored for later use both in the removal of incomplete
responses (\S \ref{sec:remove-responses}) and by constraint
verification logic (\S \ref{sec:verify-constraints}).

\subparagraph{Answer Restricts Question Answers}
\label{sec:answ-restr-quest}

The Restricting Question and Answer will be selected (first by
Question text and then by Response Value), a second question, the
Restricted Question will be selected (again, by Question Text), and a
series of Allowed Responses will be selected (by Response Values).
This information will then be stored for later use both in the removal
of incomplete responses (\S\ref{sec:remove-responses}) and by
constraint verification logic (\S \ref{sec:verify-constraints}).

\paragraph{Remove Survey Question}
\label{sec:remove-surv-quest}

A Question will be selected (by Question ID), and removed from the
storage.  All responses for the question will be removed, as will all
rules involved with the given Question.

\paragraph{Remove Question Response}
\label{sec:remove-quest-resp}

A Question Response will be selected (by Question first, then by
Response Value), and removed from the storage.  All REQUIRES and
FORBIDS rules will be removed, RESTRICTS rules that match on the
Question and Response will be removed and RESTRICTS rules that allow
the given answer will have the allowance removed.

\paragraph{Remove Question Constraint}
\label{sec:remove-quest-constr}

A Question Constraint will be selected (by Constraint ID), and removed
from storage.

\paragraph{Publish Survey}
\label{sec:publish-survey}

A Survey will be selected, and made public so that \hl{respondents} may
submit data (\S\ref{sec:data-collection}).

\subsection{Data Collection}
\label{sec:data-collection}

\subsubsection{Introduction}
\label{sec:introduction-collection}

The ability to follow a survey plan, apply the constraints, and
collect and store data from a survey is a core feature.  This will be
available to any and all respondents to submit responses.  This feature
will be further composed of several sub-features.

\subsubsection{Stimulus/Response}
\label{sec:stim-resp-collection}

This feature allows a potential respondent to access the survey through
internet address or social media. His or her answers will be recorded and
stored in a database for later access.

\subsubsection{Associated Requirements}
\label{sec:assoc-reqs-collection}

\paragraph{Begin Survey}

A possible respondent will click a link which takes them to the associated
survey. Any intro page or message chosen by survey creator will show, allowing
respondents to read any survey directions. The respondent must then click a
 ``begin'' button to confirm their desire to take the survey.

\paragraph{Verify Constraints}
\label{sec:verify-constraints}

After an answer has been submitted, the system will proceed to check
for any constraints on the following question. It will then either
present the question as is, present the question with modified response
choices, or skip to the next question (\S\ref{sec:add-quest-constr}).

\paragraph{Present Question}

A question will be presented to the respondent. After they select an answer and
click continue, the system will record it and proceed.

\paragraph{End Survey}

After the respondent completes the last question, the system will display
any post survey message implemented by the survey creator.

\subsection{Data Retrieval}
\label{sec:data-retrieval}

\subsubsection{Introduction}
\label{sec:introduction-retrieval}

The ability to retrieve the results of a survey, in one of several
possible formats, is another core feature.  This will only be
available to authorized users, and will provide output in four formats
--- \hl{SQL}, \hl{CSV}, \hl{Excel} and \hl{JSON}.  This feature will be composed of several
sub-features.


\subsubsection{Stimulus/Response}
\label{sec:stim-resp-retrieval}

Having closed a survey, an authenticated User will be able to retrieve
the data collected by the survey in various, user-selectable formats.
The data will be cleaned, possibly anonymized, and exported to the
selected format.

\subsubsection{Associated Requirements}
\label{sec:assoc-reqs-retrieval}

\paragraph{Remove Incomplete Responses}
\label{sec:remove-responses}

In order to provide a clean, complete set of Survey data authorized
users will be able to remove incomplete responses from their
Survey. Incomplete Responses are defined by answers that do not follow
the specific constraints tied to that question
(\S\ref{sec:verify-constraints}).

\paragraph{Anonymize Data}
\label{sec:anonymize-data}

Determined during survey creation (\S\ref{sec:create-survey}). Survey
designers will decide whether the survey will be anonymous or
non-anonymous. Anonymous data will have no links to any users taking
the survey. Non-anonymous surveys will, when exported, provide the
survey designer with an identifier for the respondent.

\paragraph{Export as SQL}
\label{sec:export-sql}

Survey data will be exportable as a \hl{SQL} file. This option will
provide users the option of having the data ready for use with their
own database.

\paragraph{Export as CSV}
\label{sec:export-csv}

Survey data will be exportable as a \hl{CSV} file. This option will provide
users a widespread compatibility with other applications and allows
for quick exportation and viewing.

\paragraph{Export as Excel}
\label{sec:export-excel}

Survey data will be exportable as an \hl{Excel} file. This option will give
the survey designer a fast and readable option for viewing the survey
results.

\paragraph{Export as JSON}
\label{sec:export-json}

Survey data will be exportable as a \hl{JSON} file. This option is
intended for users who want to use the data from their Survey in
programs, as JSON files are easily importable into most languages.

\section{Performance}
\label{sec:performance}

This system, given the nature, must be able to support many concurrent
users, in some situations on the order of several hundred at a time.
Through good design and tool choice, this can be assured.

\section{Logical Database Requirements}
\label{sec:logic-datab-requ}

The database used to store data for this system will be required to
provide several specific and particular capabilities, including
storage and retrieval of survey meta-data (the design of a survey,
questions, survey constraints, etc.) and the storage and retrieval of
survey responses.

\section{Design Constraints}
\label{sec:design-constraints}

As with many systems, there are several major constraints on this
application.  Highlighted here are two constraints of particular note:

\begin{enumerate}
\item For usability in academic research, the system will be required
  to be usable by those with disabilities.  This necessitates a clean
  and simple interface design that can be easily worked with a
  screen-reader.
\item For usability in academic research, the system must also be able
  to ensure that data retrieved for use may be made anonymous.  This
  in particular will be handled by a specific feature defined in
  \S\ref{sec:anonymize-data}.
\end{enumerate}

\section{Software System Attributes}
\label{sec:softw-syst-attr}

\subsection{Reliability}
\label{sec:reliability}

Defining reliability as probability of operation without any system
errors (those not caused by users or circumstances beyond system
control), the software should be $95\%$ reliable.

\subsection{Availability}
\label{sec:availability}

This software should be available $90\%$ of the time, barring
catastrophic events outside of the control of the developers (hosting
issues, for instance).

\subsection{Security}
\label{sec:security}

To ensure that the system is secure, only certain classes of users may
be able to access certain functions, in particular, only authenticated
users may be able to either design surveys
(\S\ref{sec:survey-designer}) or retrieve survey data
(\S\ref{sec:data-retrieval}).  This will be handled through the use of
Role-Based Access Control and data will be stored and maintained
through the use of database internal authentication.

\subsection{Maintainability}
\label{sec:maintainability}

To ensure that the software is maintainable, the design will be
composed using techniques such as object-oriented design and
interface-based design.

\subsection{Portability}
\label{sec:portability}

To ensure that the software is portable, the Python programming
language will be used, with code written to conform with PEP~8.
For database access, a database-agnostic library will be used, and for
interfacing with webservers, a webserver interface library (such as
WSGI or CGI) will be used.

\appendix

% Add any appendices here, if so desired

\end{document}

%%% Local Variables:
%%% mode: latex
%%% TeX-master: t
%%% End:
